\documentclass[10pt,letterpaper]{article}
\usepackage[margin=1in]{geometry}
\usepackage{setspace}
\usepackage{fancyhdr}
\usepackage{lastpage}

% Put watermark on
% \usepackage{draftwatermark}
% \SetWatermarkText{Draft}
% \SetWatermarkScale{7}
% \pagestyle{fancyplain}

\lhead{}
\chead{Central Massachusetts Amateur Radio Association}
\rhead{}
\lfoot{}
\cfoot{}
\rfoot{Page \thepage\ of \pageref{LastPage}}

\begin{document}

\begin{center}
{\huge January 2016 Board of Directors Meeting}\\
\emph{of the}\\
{\Large Central Massachusetts Amateur Radio Association}\\
\emph{Submitted by Mark Ide \texttt{W1IDE}, Secretary}
\end{center}

\section{Meeting Called to Order}
The CMARA January 2016 board of directors meeting was called to order on January 21, 2016 at 9:17 PM by CMARA president Bob Peloquin (\texttt{KB1VUA}).

\section{Attendance}

\subsection{Officers Present}

\begin{tabular}{|l|l|l|c|}
  \hline
  \textbf{Position} & \textbf{Name}  & \textbf{Callsign} & \textbf{Present} \\ \hline
  President         & Bob Peloquin   & \texttt{KB1VUA}   & Yes              \\
  Vice President    & Art Kass       & \texttt{WA1RCQ}   & Yes              \\
  Secretary         & Mark Ide       & \texttt{W1IDE}    & Yes              \\
  Treasurer         & Jim Singer     & \texttt{N1EKO}    & Yes              \\
  Webmaster         & Lynn Glagowski & \texttt{WB1CCL}   & No               \\
  \hline
\end{tabular}

\subsection{Board of Directors Present}

\begin{tabular}{|l|l|c|}
  \hline
  \textbf{Name}     & \textbf{Callsign} & \textbf{Present} \\ \hline
  Adrian Zeffert    & \texttt{AB2IX}    & Yes              \\
  Harold Carlson    & \texttt{N1ZC}     & No               \\
  Dick Jubinville   & \texttt{W1REJ}    & Yes              \\
  Terry Glagowski   & \texttt{W1TR}     & Yes              \\
  Randy Dore        & \texttt{W4FEB}    & Yes              \\
  Johnathan Sherman & \texttt{WW2JS}    & No               \\
  \hline
\end{tabular}

\subsection{Members Present}
\begin{itemize}
\item Greg (\texttt{WA1JXR})
\item Don (\texttt{W3DEC})
\end{itemize}

\subsection{Guests \& Visitors}

No guests were present.

\section{Old Business}

There was no old business tabled.

\section{New Business}

\begin{enumerate}
\item Art (\texttt{WA1RCQ}) brought up the concerns regarding field day and the securing the site.
\begin{itemize}
\item We are no longer directly working with David Prouty High School when we secure the site, we are now working with the school district and they tend to have less flexibility.
\item We need to retrieve our proof of insurance in order to secure the site.
\item We need to increase our insurance from ``\$1 million - \$2 million'' to ``\$1 million - \$3 million''.
\item We need to request a certificate of insurance listing David Prouty as the holder for three days. Bob (\texttt{KB1VUA}) mentioned this is fairly common in towns.
\item If we stay with Moose Hill, we may have to pay \$36 per hour to use the field house. If we can't or don't want to take on that cost, we'll have to form a backup plan for sanitary facilities. We need to consider if it's cheaper to rent porta-potties or to rent the field house.
\item We need to get the proof of our non-profit status (501c). Bob (\texttt{KB1VUA}) said this can be found on the Secretary of State website.
\item We should start to consider backup locations. A few locations that were thrown out were Bob (\texttt{KB1VUA})'s astronomy club's field and the Mercy Centre.
\end{itemize}

\item The club does not have a presentation lined up for the February meeting. Some of the following were ideas thrown out:
\begin{itemize}
\item Mark (\texttt{W1IDE}) - Linux
\item Terry (\texttt{W1TR}) - Tower \& Generator
\item Adrian (\texttt{AB2IX}) - How to Use Test Equipment
\item Tom (\texttt{NE1R}) - Knots
\item Don (\texttt{W3DEC}) - Near Vertical Incidence Skywave
\end{itemize}
It was decided by unanimous decision that Don (\texttt{W3DEC}) will give his presentation on Near Vertical Incidence Skywave (NVIS) in February.

\item Earlier in the evening, Don (\texttt{W3DEC}) made a movement to discuss purchasing new repeater equipment for no more than \$1,500.00.
\begin{itemize}
\item We have a fair amount of money from the generosity of a ham; we have \$1,400 in the memorial account and \$900 in the repeater fund right now.
\item How much did we and will we spend on field day? And that's for one day. We should consider servicing the unit that the club uses every day.
\item The current repeater is a 45 year old Micor model.
\item Don (\texttt{W3DEC}) has observed an uneven and inconsistent footprint. He could get 9 S-Units one day, and 4 S-Units the next.
\begin{itemize}
\item It's possible the antenna is going bad. We were originally transmitting on a StationMaster antenna that was in the middle of the tower, but that failed. We're now transmitting on a folded dipole that's on one leg of the tower. It's possible it's on the wrong leg.
\item Two-Meter propagations do change regularly
\item If the old radio tower sways, that would change propagation patterns too.
\end{itemize}
\item Greg (\texttt{WA1JXR}) stated that ``the repeater is stable as rock'' and that they've watched it.
\item Greg (\texttt{WA1JXR}) clarified about the state and configuration of our antennas. We have two antennas on the tower. The receive antenna is at the very top of the tower and is shared with other tenants. The transmitting antenna is a folded dipole that's on one leg of the tower and it's about half way up the tower; we are the only ones using the transmitting antenna.
\item Regarding a backup solution, Greg (\texttt{WA1JXR}) has spare parts for the existing Micor repeater and he also has a different repeater that we could put into service if the need arose. It's also worth mentioning that we can also order another repeater at any point and install it with short notice.
\item Should we buy another antenna to replace the existing one?
\begin{itemize}
\item Terry (\texttt{W1TR}) will ask his other club about which antennas they would suggest for our situation.
\item We may want to consider measuring the footprint. Take a couple measurements across various locations and actually base our decisions off numbers.
\item We may invite Kurt for input on the antenna selection as he may have opinions.
\end{itemize}
\item No action has been determined nor is recommended at this time.
\end{itemize}
\end{enumerate}

\section{Next Meeting}

The next meeting will be Thursday, February 18, 2016 at the Oakdale United Methodist Church, 15 North Main Street, West Boylston, MA 01583.\\

\section{Meeting Adjourned}
The meeting was adjourned at 9:58 PM by CMARA president Bob Peloquin (\texttt{KB1VUA}).

\end{document}
